\capitulo{7}{Conclusiones y Líneas de trabajo futuras}

\section{Conclusiones}
Tras finalizar el proyecto se puede concluir que desarrollar una aplicación web con Vaadin, puede resultar bastante sencillo, sobre todo cuando apenas se tienen conocimientos sobre lenguaje web. Poder trabajar con Java, que es un lenguaje muy extendido, y con tecnologías como Spring Boot que facilitan la configuración de otras dependencias es una de las grandes ventajas aportadas por esta plataforma.

Por el contrario, un punto negativo que tiene esta plataforma es el numeroso lanzamiento de versiones y el mantenimiento a largo plazo de muchas de ellas. Debido a esto, y a que no queda muy claro en su documentación qué versión es ventajosa para ciertas cosas frente a otras, resulta una tarea difícil elegir la versión adecuada desde el principio. Además, en un TFG en el que tampoco se dispone de un largo periodo para el desarrollo, llegado cierto punto es inconcebible hacer el cambio de versión.

La curva de aprendizaje de Vaadin resultó elevada al principio, ya que al integrar otras tecnologías como Spring Boot y tratarse de algo bastante nuevo para mí, saber configurar bien cada cosa no fue sencillo. Pero cabe destacar que en su página disponen de gran cantidad de tutoriales y explicaciones, por lo que superado el pequeño bache del inicio y pese a la difícil tarea de intentar integrar las herramientas de Google, el resto fue rodado.

A pesar de que el Trabajo de \textit{Fin} de Carrera se corresponde a 12 créditos, unas 300 horas de trabajo, un desarrollo web integrando varias herramientas externas conlleva muchísimo más tiempo, y más cuando apenas se ha tratado este tema. De hecho ha sido difícil compatibilizarlo con la realización de las prácticas curriculares y extracurriculares, pues a partir de febrero sólo disponía de la mitad de tiempo.

El presente trabajo ha servido para aplicar muchos de los conocimientos aprendidos a lo largo de la carrera, y en el periodo de prácticas, en cuanto a trabajo relacionado con la base de datos, desarrollo de código eficiente, intentando evitar la duplicación agrupándolo en clases comunes e implementación de una interfaz gráfica lo más amigable y visualmente agradable posible. Lo cual se puede considerar la finalidad última de un TFG.

\section{Líneas de trabajo futuras}
A pesar de entregar un producto completamente funcional que cumple con el objetivo de la reserva de aulas, se pueden integrar otras funcionalidades al mismo para completarlo:

\begin{itemize}
    \item Enviar un correo de confirmación tras la realización o modificación de una reserva.
    \item Cuando las nuevas versiones de Vaadin lo permitan, integrar la API de Google Calendar y el componente de Google Sign-In para poder insertar los eventos en el calendario personal del usuario responsable.
    \item Mostrar el calendario con las reservas en una interfaz modo agenda, en la ventana de Consulta de Reservas para que resulte más visual.
    \item Incluir la internacionalización de la aplicación.
\end{itemize}